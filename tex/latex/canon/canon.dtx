% \iffalse meta-comment
%
% \fi
%% \CharacterTable
%%  {Upper-case    \A\B\C\D\E\F\G\H\I\J\K\L\M\N\O\P\Q\R\S\T\U\V\W\X\Y\Z
%%   Lower-case    \a\b\c\d\e\f\g\h\i\j\k\l\m\n\o\p\q\r\s\t\u\v\w\x\y\z
%%   Digits        \0\1\2\3\4\5\6\7\8\9
%%   Exclamation   \!     Double quote  \"     Hash (number) \#
%%   Dollar        \$     Percent       \%     Ampersand     \&
%%   Acute accent  \'     Left paren    \(     Right paren   \)
%%   Asterisk      \*     Plus          \+     Comma         \,
%%   Minus         \-     Point         \.     Solidus       \/
%%   Colon         \:     Semicolon     \;     Less than     \<
%%   Equals        \=     Greater than  \>     Question mark \?
%%   Commercial at \@     Left bracket  \[     Backslash     \\
%%   Right bracket \]     Circumflex    \^     Underscore    \_
%%   Grave accent  \`     Left brace    \{     Vertical bar  \|
%%   Right brace   \}     Tilde         \~}
%
% \iffalse
%    \begin{macrocode}
%<package>\NeedsTeXFormat{LaTeX2e}[1995/12/01]
%    \end{macrocode}
%
%    \begin{macrocode}
%<package>\ProvidesPackage{canon}
%<*driver>
\ProvidesFile{canon.drv}
%</driver>
   [2012/02/02 v1.0 style file ``canon.sty''
%<package>    by KIEDA Yuwsuke]
%    \end{macrocode}
%
% \section{使用するdriverファイル}
%    \begin{macrocode}
%<*driver>
]
\documentclass{jltxdoc}
%    \end{macrocode}
%    \begin{macrocode}
\EnableCrossrefs
\setcounter{StandardModuleDepth}{1}
\GetFileInfo{canon.drv}
%    \end{macrocode}
%    \begin{macrocode}
\begin{document}
  \DocInput{canon.dtx}
% ^^A\PrintIndex
% ^^A\PrintChanges
\end{document}
%</driver>
%    \end{macrocode}
% \fi
% \title{\textsf{canon.sty \fileversion{}}
%  (document version \textsf{v1.0})}
% \author{%
% Copyright 2012 by KIEDA Yuwsuke
% }
% \date{\filedate}
% \maketitle
% \tableofcontents
% \MakeShortVerb{\|}
% \StopEventually{} ^^A
%
% \section{序}
% 本スタイルファイル\textsf{canon.sty}は,
% Jan Tschicholdのいう古典的な欧文組版の版面の
% “Van de Graaf canon”を実現するものである.
% 固定長は二つ,|\headheight| (12\,pt)と
% text blockとmargin columnの間隔(30\,pt)である.
% \section{パッケージ・オプション}
% \begin{macro}{a3paper}
%    \begin{macrocode}
\DeclareOption{a3paper}{%
  \paperwidth  297 true mm
  \paperheight 420 true mm}
%    \end{macrocode}
% \end{macro}
% \begin{macro}{a4paper}
%    \begin{macrocode}
\DeclareOption{a4paper}{%
  \paperwidth  210 true mm
  \paperheight 297 true mm}
%    \end{macrocode}
% \end{macro}
% \begin{macro}{a5paper}
%    \begin{macrocode}
\DeclareOption{a5paper}{%
  \paperwidth  148 true mm
  \paperheight 210 true mm}
%    \end{macrocode}
% \end{macro}
% \begin{macro}{a6paper}
%    \begin{macrocode}
\DeclareOption{a6paper}{%
  \paperwidth  105 true mm
  \paperheight 148 true mm}
%    \end{macrocode}
% \end{macro}
% \begin{macro}{jisb3paper}
%    \begin{macrocode}
\DeclareOption{jisb3paper}{%
  \paperwidth  364 true mm
  \paperheight 515 true mm}
%    \end{macrocode}
% \end{macro}
% \begin{macro}{jisb4paper}
%    \begin{macrocode}
\DeclareOption{jisb4paper}{%
  \paperwidth  257 true mm
  \paperheight 364 true mm}
%    \end{macrocode}
% \end{macro}
% \begin{macro}{jisb5paper}
%    \begin{macrocode}
\DeclareOption{jisb5paper}{%
  \paperwidth  182 true mm
  \paperheight 257 true mm}
%    \end{macrocode}
% \end{macro}
% \begin{macro}{jisb5paper}
%    \begin{macrocode}
\DeclareOption{jisb6paper}{%
  \paperwidth  128 true mm
  \paperheight 182 true mm}
%    \end{macrocode}
% \end{macro}

% \begin{macro}{6/9}
% オプション|6/9|によって,canonを実現する(デフォルト).
%    \begin{macrocode}
\newif\if@six@f@nine \@six@f@ninefalse
\DeclareOption{6/9}{\@six@f@ninetrue}
%    \end{macrocode}
% \end{macro}
% \begin{macro}{7/10}
% オプション|7/10|は$6/9$ではなく$7/10$で版面設計を行う.
%    \begin{macrocode}
\newif\if@seven@f@ten \@seven@f@tenfalse
\DeclareOption{7/10}{\@six@f@ninefalse%
  \@seven@f@tentrue}
%    \end{macrocode}
% \end{macro}
% \begin{macro}{paperwidth}
% |\paperwidth|を|\textheight|に設定し,
% 比を保って|\textwidth|を設定する.
%    \begin{macrocode}
\newcount\@sqrt@two@M \@sqrt@two@M = 14142
\newif\if@paperwidth@textwidth \@paperwidth@textwidthfalse
\DeclareOption{paperwidth}{\@six@f@ninefalse%
  \@paperwidth@textwidthtrue}
%    \end{macrocode}
% \end{macro}

% \section{デフォルト・オプション}
%    \begin{macrocode}
\ExecuteOptions{6/9}
\ProcessOptions
%    \end{macrocode}

% \section{版面設計}
% \subsection{天マージン}
%    \begin{macrocode}
\headheight 12 true pt
\maxdepth .5\topskip
\@tempdima\paperheight
\if@six@f@nine
  \divide\@tempdima by 9\relax
\fi
\if@seven@f@ten
  \divide\@tempdima by 10\relax
\fi
\if@paperwidth@textwidth
  \advance\@tempdima -\paperwidth
  \divide\@tempdima by 3\relax
\fi
\topmargin\@tempdima
\advance\topmargin-\headheight
\advance\topmargin-\headsep
\advance\topmargin-1 true in
%    \end{macrocode}

% \subsection{text block width}
%    \begin{macrocode}
\@tempdima\paperwidth
\if@six@f@nine
  \divide\@tempdima by 9\relax
  \textwidth 6\@tempdima
\fi
\if@seven@f@ten
  \textwidth .7\@tempdima
\fi
\if@paperwidth@textwidth
  \divide\@tempdima\@sqrt@two@M
  \multiply\@tempdima\@M
  \textwidth\@tempdima
\fi
\@settopoint\textwidth
%    \end{macrocode}

% \subsection{text block height}
%    \begin{macrocode}
\@tempdima\paperheight
\if@six@f@nine
  \divide\@tempdima by 9\relax
  \@tempdima 6\@tempdima
\fi
\if@seven@f@ten
  \@tempdima .7\@tempdima
\fi
\if@paperwidth@textwidth
  \@tempdima\paperwidth
\fi
\advance\@tempdima -\topskip
\advance\@tempdima 5 true pt
\divide\@tempdima \baselineskip
\@tempcnta\@tempdima
\textheight\@tempcnta \baselineskip
\advance\textheight \topskip
%    \end{macrocode}

% \subsection{side margins}
%    \begin{macrocode}
\@tempdima\paperwidth
\if@six@f@nine
  \divide\@tempdima by 9\relax
\fi
\if@seven@f@ten
  \divide\@tempdima by 10\relax
\fi
\if@paperwidth@textwidth
  \advance\@tempdima -\textwidth
  \divide\@tempdima by 3\relax
\fi
\evensidemargin 2\@tempdima
\oddsidemargin \@tempdima
\advance\evensidemargin -1 true in
\advance\oddsidemargin -1 true in

\@tempdima\paperwidth
  \advance\@tempdima -\textwidth
  \advance\@tempdima -\oddsidemargin
  \advance\@tempdima -\marginparsep
  \advance\@tempdima -30 true pt
\marginparwidth\@tempdima
\@settopoint\marginparwidth
%    \end{macrocode}
%
% \Finale
%
\endinput
